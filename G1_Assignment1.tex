\documentclass{article}
\usepackage[utf8]{inputenc}
\usepackage[danish]{babel}
\usepackage{graphicx}
\font\myfont=cmr12 at 10pt

\title{\huge Software implementation of a personal ECG-scanner \\ \LARGE Assignment 1 \\ \LARGE 02131 Indlejrede Systemer}
\author{Gruppe 1: \\Anna Ølgaard Nielsen, s144437 \\ Van Anh Thi Trinh, s144449 \\ Martin Dariush R. Hansen, s144459}

\usepackage[ddmmyyyy]{datetime}
\newdate{date}{29}{09}{2015}
\date{\myfont \displaydate{date}}

\begin{document}
\maketitle

\newpage

\section*{Abstract}

\tableofcontents

\newpage
\section{Introduktion}
Vi er tre bachelorstuderende fra softwareteknologi på DTU på tredje semester, der i vores første ud af tre projekter i kurset, 02131 indlejrede systemer, har arbejdet med data fra en ECG Scanner i form af en simuleret hjerterytme. I forbindelse med forløbet, er vi blevet undervist i c, som vi derudfra har anvendt til at skrive et program, der behandler denne data. De overordnede krav til programmets funktionalitet blev givet på forhånd, og der vil i denne rapport fokuseres på tankerne bag programmet og kravene dertil, via en analyse, design og implementering samt slutresultatet med en tilhørende diskussion og konklusion.


\newpage
\section{Arbejdsmetode og samarbejde}
Der er ofte mange måder at løse forskellige problemer på. Vi har i samarbejde taklet disse problemer ved først at lave en brainstorm over mulige løsninger samt deres fordele og ulemper. Derudfra har vi valgt de løsninger, som vi vurderede løste problemet bedst.

Til start udviklede vi en samarbejdskontrakt, hvor bl.a. vores ambitioner og samarbejdspolitikker blev formuleret. Vi har arbejdet sammen i de lektioner, vi har fået til rådighed på DTU, og har flere gange mødtes på andre tidspunkter og arbejdet sammen. En del af arbejdet er blevet delt op imellem os, hvoraf vi hver i sær har arbejdet hjemmefra. For at udnytte gruppens ressourcer så godt som muligt, har vi alle lavet noget samtidigt i form af individuel kodeskrivning, pair-programming, individuel rapportskrivning eller gruppediskussioner.

Vi har været gode til at overholde vores aftaler, og der har ikke været problemer med gruppesamarbejdet. Selve koden er skrevet i eclipse og delt via git, mens rapporten er skrevet og delt via Google Docs og kompileret med Latex.

\section{Analyse}
\subsection{Problemstilling}
For mennesker med hjerteproblemer, er det vigtigt at få den rette hjælp, da sådanne problemer i nogle tilfælde kan medfører hjertestop, hvoraf personen kan miste livet. Disse hjerteproblemer opstår ofte uregelmæssigt, og kan derfor være svære at detektere hos en læge eller på et sygehus, hvor man vil opholde sig i begrænsede mængder af tid.

En ECG Scanner måler elektrisk aktivitet og bruges til at måle en persons hjerterytme. Scanneren kan anvendes over længere tidsforløb og er mobil, hvilket gør, at man ikke konstant skal være under opsyn for at kunne få en diagnose på hjerteproblemerne og hjælp, hvis man skulle få hjertestop.

ECG Scanneren i sig selv kan dog ikke hjælpe meget uden den rette software til at indlæse dataen, behandle og tolke den. Softwaren skal derudover også kunne informere brugeren om hjertets tilstand. En tilsynsførende skal også kunne blive alarmeret, hvis der er noget galt. Det er vigtigt, at softwaren gør alt dette i realtid og f.eks. ikke alarmerer lægen for sent. Derudover skal den tilsynsførende kunne se, om der har været uregelmæssigheder fra besøg til besøg.

Da programmet skal kunne køre på en chip med begrænset hardware og batterilevetid, må det hverken bruge for meget plads eller regnekraft og deraf strøm. Dertil bør der heller ikke blive anvendt datatyper som floats, som bruger relativt meget plads, og ikke er direkte understøttet af alt hardware.

Sidst men ikke mindst er det vigtigt at nævne, at softwaren skal være pålidelig og stabil, og altså altid fungere på samme måde og ikke pludselig fejle.

\subsection{Funktioner}
For at løse ovenstående problemstilling, skal der laves et program. Kravene til programmets funktionalitet beskrives i de følgende afsnit.

\subsubsection{Afgrænsning}
I den virkelige verden, ville programmet skulle indlæse data fra ECG Scanneren i realtid. I vores tilfælde skal programmet dog blot indlæse data fra en tekstfil, som er blevet udleveret på forhånd. Dog skal programmet ikke indlæse flere datasamples ad gangen, men skal indlæse og behandle et element ad gangen, inden ny data indlæses.

Det ville normalt også være et krav at den tilsynsførende og ikke kun brugeren ville blive alarmeret, hvis der skulle opstå problemer med hjerterytmen, eller at en ambulance evt. automatisk ville blive sendt ud til éns lokation. I dette projekt 

\subsubsection{Datastrukturer}
En ECG Scanner er et indlejret system, der skal være så effektiv som muligt på begrænset plads. Da et program til en ECG Scanner kan udformes på mange forskellige måder, har vi gjort os en del overvejelser om hvilken løsning, der passer bedst i denne situation.

Først skulle vi finde den ideelle datastruktur at gemme de enkelte samples ned i. Vi overvejede: en kø, et dynamisk array, en hægtet liste og et begrænset array.

Fordelene ved at benytte en kø, er at det ville være nemt og hurtigt at tilgå den første og den sidste værdi i køen. Normalt er en kø en liste, som bruger ubegrænset plads, men det ville være muligt at implementere den i et begrænset array som en torus. Altså når køen når til den sidste plads i arrayet, hopper den frem til den forreste plads i arrayet og fortsætter derfra. For at det skulle virke, skulle vi fjerne elementer i takt med at der blev indsat elementer, så man kun holdt på de seneste samples. Desværre kan man ikke læse andre samples i arrayet udover den senest og den tidligst indsatte. For at kunne få en bestemt sample, ville man skulle fjerne andre samples indtil den bestemte sample lå som den tidligste, og det ville ikke være optimalt, da nogle af vores funktioner kræver man kan få fat i bestemte samples uden at skulle fjerne andre samples.

Et andet alternativ er at bruge et dynamisk array, der også fungerer som en liste. Et dynamisk array laver et array af en bestemt længde, og når man har fyldt det ud, opretter den et nyt array der er dobbelt så stort, flytter alle elementer over i det nye array og sletter det gamle array. Derved bruger et dynamisk array ubegrænset plads og bruger forholdsvis lang tid på at oprette nye arrays og flytte alle elementer. En løsning på det, ville være at man efter noget tid nulstillede det dynamiske array, så det ikke brugte for meget plads, men så ville man skulle smide alle samples ud og starte forfra. Man kunne også slette gamle samples, når man indsatte nye samples, og derved holde konstant plads, men så kunne man ligeså godt bare bruge en begrænset array.
Et tredje alternativ ville være at bruge en hægtet liste, da det er nemt at indsætte og slette fra den uden at skulle rykke alle samples. F.eks. hvis man sletter en sample midt i den hægtede liste, så ville det dynamiske array bruge tid på at flytte alle de andre samples frem, så der ikke var “huller” midt i listen. Desværre tager det rigtig lang tid at søge en hægtet liste igennem for et bestemt element. En ide var, at sætte pointers på alle elementerne, så man hurtigt kunne få værdierne uden at skulle lede listen igennem, og hver gang et element blev slettet eller indsat bliver alle pointers rykket. Det ville så bare gøre, at listen skulle være begrænset i plads, så man ikke oprettede alt for mange pointers. Igen ville dette kunne løses på samme måde med et begrænset array.

Det sidste alternativ, som også er den løsning vi har valgt at arbejde videre med, er et begrænset array. Man kan hurtigt få adgang til alle værdier, og det er hurtigt at indsætte og slette. Desværre kan man kun holde en bestemt mængde data, så man ikke kan gå tilbage og se data fra langt tilbage. Dette er en mindre detalje, da vi kun skal bruge få data at beregne filtre med, og derved ikke har brug for en lang historik af samples. 
Vi har vurderet at et begrænset array er mest ideelt at bruge i denne situation, da det er hurtigt og ikke fylder så meget.

\subsubsection{Samples og tidsmåling}
Inputtet, som softwaren modtager fra ECG Scanneren, består af individuelle målinger (samples) for den målte elektriske aktivitet (styrke) af hjertet [ECG]. ECG Scanneren giver 250 samples som output per sekund. Altså er frekvensen:

Hver gang en sample indlæses, forøges en optællings-integer med én. På denne måde kan en sample betragtes som et punkt med en bestemt styrke til en given tid. Idet integers ikke kan indeholde uendeligt store tal, vil denne optæller på et tidspunkt gå i overløb. Det antages at den anvendte chip kan anvende 32-bit integers. Heraf vil der gå så meget tid inden variablen vil have overløb, hvis det antages, at det er en unsigned int:

Det vurderes, at 200 dage er en tilstrækkelig levetid for programmet, idet det antages at programmet enten genstartes, når batteriet skal oplades/udskiftes, eller ved hvert lægecheck. Alternativt kunne man sætte værdien til 0, inden overløbet finder sted.

\subsubsection{Indlæsning af Data}
Det er et krav, at man kun får en sample at arbejde med ad gangen, da det er sådan det kommer til at foregå på en rigtig patient. Der er givet en tekstfil med alle målingerne, men der skal altså kun indlæses én måling ad gangen. Hver gang en måling indlæses, indlæses værdien på linjen i tekstfilen, som patienten er nået til. For at læse en bestemt linje er scanneren, som indlæser værdier fra filen, nødt til at læse alle linjerne ned til den ønskede linje. Da dette tager længere tid jo flere målinger der har været, indlæses alle værdierne i stedet med det samme og gemmes i censoren. Hver gang funktionen getNextData kaldes fra main-funktionen, returneres dog kun én værdi og det virker derfor stadig som om at censoren kun har indlæst en enkelt ny værdi som returneres til main-funktionen, der derefter kan sende værdien til filtrering.

\subsubsection{Filtre}
Dataen behandles via filtre. Low og high-pass filtre fjerner støj i områder i dataen som på baggrund af frekvens eller amplitude adskiller sig fra, hvordan hjerteslag normalt ser ud i dataen. Andre filtre gør det nemmere for programmet at identificere hjerteslagene ved at forstærke og udglatte dataen. Filtrenes formler er givet på forhånd i deres rene (ikkeimplementerede) form, og formlerne udledes eller forklares ikke i denne rapport.

\subsubsection{Peakdetektion}
Efter filtreringen detekteres højdepunkterne (peaks) i dataen, hvormed programmet vurderer om højdepunkterne kan betragtes som hjerteslag eller ej. Et forslag til en måde at gøre dette på er givet på forhånd i form at et flowchart. Flowchartet giver også et forslag til, hvordan programmet kan afgøre, om der er problemer med hjerterytmen eller om der er hjertestop. Disse forslag anvendes i vores program.

\subsubsection{Output}
Når en R-Peak er detekteret, skal patienten kunne se det med det samme, med oplysning om intensiteten af den fundne R-Peak og på hvilket tidspunkt den blev detekteret. Derudover skal patienten også kunne se hvor lang tid der er gået fra den sidst-detekterede R-Peak - RR-intervallet. Patienten skal også kunne se sin puls og advares hvis en R-Peak har en intensitet på under 2000 eller hvis der i træk har været fem RR-intervaller, som ikke har ligget inden for et bestemt interval.

\subsubsection{Programanalyse og ydeevne}

\section{Design}
Vores program består af en main.c og tre andre c-filer, Peaks.c, Filter.c og Sensor.c, som main.c har adgang til via deres header-filer. 

\subsection{Sensor}
I Sensor.c opretter vi et file-objekt, der læser en sample ad gangen fra den udleverede ECG.txt fil, der indeholder simulerede data fra en sensor. Hvis næste sample ikke er tom, bliver der scannet et nyt sample ind som main.c får adgang til, ved at kalde getNextData() funktionen.

\subsection{Filtre}
Filter.c består af 5 lag filtre, Low-pass, High-pass, Derivative, Squaring og Moving Window Integration. Det fungerer på den måde, at Low-pass henter rådata fra main.c og giver resultaterne videre til High-pass. High-pass læser resultaterne fra Low-pass ind og giver resultaterne videre til Derivative filteret. Dette fortsætter indtil det bestemte sample er kommet igennem alle 5 filtre, hvorefter det bliver sendt tilbage til main.c, der printer det som en del af outputtet.

I hver af formlerne beskriver x inputtet i den aktuelle funktion, og y beskriver den resulterende værdi, efter det bestemte sample er kørt igennem filteret.

Low-pass filteret:
 
Low-pass filteret filtrerer alle frekvenser over 12 Hz fra og lader kun de lave frekvenser komme igennem.

High-pass filteret:

High-pass filteret filtrerer alle frekvenser under 6 Hz fra og lader kun de høje frekvenser passere. På denne måde har vi kun målinger mellem 6 og 12 Hz tilbage, som bliver sendt videre til derivative filteret.

Derivative filteret:

Derivative filteret forstærker signalerne, så det er nemmere at detektere peaks senere hen. Desuden flytter den hele grafen til at ligge ved y = 0, når der ikke er detekteret nogle udslag.

Squaring filteret:
Squaring filteret kvadrerer inputtet, af den grund at små forskelle bliver større og nemmere at se, og alle signaler bliver positive. Det gør det også nemmere at detektere peaks senere, så man ikke skal tage højde for negative peaks.

Moving Window Integration filteret:

Moving Window Integration filteret gør signalet mere jævn. Det er en fordel, da man bedre kan se, hvordan hjerterytmen slår og bevæger sig. Det er angivet at N har værdi 30.

\subsection{Peaks}
Den generelle tanke ved at detektere peaks er, at man finder et toppunkt: Dette gør så at vi ikke detektere “bløde” peaks, hvor følgende ville gælde:

For at detektere peaks har vi brugt QRS-algoritmen, som ses som flowchart nedenfor:

Man finder først en peak, hvor det gælder at Derefter tjekker vi om peaket er større end threshold 1, som er en threshold der adskiller noise fra peaks. Hvis peak’en ikke er større end threshold 1, må det ikke være en R-peak med bare noget støj. Derfor bliver NPKF reguleret, da den holder støj-værdien. Efterfølgende bliver både threshold 1 og 2 reguleret ud fra NPKF. Hvis det detekteret peak er større end threshold 1 må det være en R-peak. Vi udregner afstanden fra denne R-peak til den forrige R-peak, og tjekker om den ligger inden for intervallet mellem og . Hvis dette er tilfældet er det en normal R-peak, og alle variable bliver reguleret efter denne R-peak. Hvis R-peaken ikke ligger inden for intervallet, tjekker vi om dens interval er større end , der så kan indikere om der måske er hjertestop. Hvis intervallet for R-peaken ikke er større end bliver den bare ikke brugt til noget, og algoritmen går i gang med næste detekteret peak. Hvis den interval derimod er større end, så søger den igennem de seneste peaks, og tjekker om en af dem er større end threshold 2, og hvis den er bliver den peak kategoriseret som en unormal R-peak. På den måde kan vi både detektere hjertestop og uregelmæssigheder i ens hjerterytme. 
Der udstedes en advarsel til brugeren, hvis 5 R-peaks i træk overstiger, og hvis der detekteres en peak med en værdi mindre end 2000.


\section{Implementering}

\subsection{Indlæsning af data}
Målingerne læses i filen ’sencor’, som indlæser en tekstfil med én måling pr. linje. Da ’sencor’ skal fungere, som en sensor, som måler én måling ad gangen, kan alt data ikke indlæses på samme tid, men skal indlæses løbende. Derfor indlæses hele tekstfilen i stedet for i et array med én værdi pr. indeks, men når funktionen getNextData i ’sencor’  kaldes, må kun én værdi gives videre. En tæller holder derfor styr på nummeret på målingen, som patienten er nået til, så hver gang getNextData kaldes, returneres den værdi i arrayet, som svarer til det, som tælleren angiver. Derefter vokser tælleren med én, så målingen fra næste indeks i arrayet kan returneres næste gang getNextData kaldes. Denne måling gemmes i et array ’data’, som indeholder de nyeste målinger. Målingen filtreres derefter før den næste måling hentes.

\subsection{Filtrering}
Filtreringen af den målte data foregår i en separat fil med en funktion for hver filtrering. Til hver filtrering er også oprettet et array, som skal holde på de nyeste målinger, som er filtreret af det tilhørende filter – f.eks. holder ’lowPass’-arrayet på den nyeste data, der er blevet filtreret gennem Low-pass-filtret. Hvert array har forskellig længde afhængig af hvor mange værdier det næste filter har brug for. For at filtrere en værdi i Derivative -filteret, er der f.eks. kun brug for den nyeste måling og 2., 4., og 5. nyeste måling. Derfor behøver filtreringen før kun at gemme de fem nyeste målinger, og arrayet har derfor længden 5.
Hver filter-funktion bliver kaldt med sit eget tilhørende array som parameter, arrayet tilhørende filteret lige før den, de to arrays længder og nummeret på den nyeste måling – ’lowpass’-funktionen, der tager imod den rå data, får udover ’lowpass’-arrayet, arrayet ’data’ i stedet for et array med filtreret data. I filter-funktionerne bruges data fra arrayet tilhørende den tidligere filtrering til at beregne ny data, som gemmes i dens eget array. Når et array er fyldt ud, overskrives de gamle værdier blot med de nye værdier med start i indeks 0 igen. Arraylængderne bruges her til at sikre at værdierne ligger i den rigtige rækkefølge. F.eks. sikres der, at når værdien før værdien i indeks 0 skal bruges – dvs. uden for arrayet -  hentes den sidste værdi i arrayet i stedet.

\subsection{Peak Detektion}
Peak detection foregår også i en separat fil ’peaks’ med de to funktioner ’findPeaks’ og ’findRPeaks’. Fra main-funktionen kaldes efter filtreringen på ’findPeaks’, der som navnet siger, finder peaks. Funktionen tager som parametre den helt filtrerede data i et array ’mwi’, nummeret på målingen, et array til tiden (nummeret på målingen) og et array til ‘rPeaks’. ’mwi’ indeholder de tre nyeste filtrerede målinger, og med nummeret på målingen kan funktionen beregne om den nyeste måling ligger i indeks 0, 1 eller 2. Når denne er fundet, findes målingen, lige før den, og der tjekkes om denne er større end de to andre værdier. Hvis den er større, er dette et peak og den gemmes i ’peaks’-arrayet, mens målingens nummer gemmes i ’time’-arrayet. Der kaldes derefter på funktionen ’findRPeaks’, der som parametre tager ’time’-arrayet og et array til R-Peaks ’rPeaks’.
I ’findRPeaks’ undersøges først om den nyeste peak i ’peaks’-arrayet er større end grænsen ’threshold1’. Er dette tilfældet undersøges der om RR-intervallet – der er tiden fra sidste fundne R-Peak til nyeste fundne peak - ligger mellem værdierne rrLow og rrHigh. Hvis den gør det, er den nyeste fundne peak en R-Peak, og den gemmes i ’rPeaks’-arrayet. Alle værdierne, der bruges til at finde R-Peaks opdateres derefter som angivet i Design-afsnittet. Hvis RR-intervallet ikke ligger mellem ’rrLow ’ og ’rrHigh ’, undersøges i stedet om den er større end værdien ’rrMiss’. Er den det, søges baglæns i ’peaks’-arrayet fra den nyeste peak for at finde en peak der er større end ’threshold2’, der så er en R-Peak, som skal gemmes i ’rPeaks’-arrayet. Igen opdateres værdierne, som bruges til at finde R-Peaks. Er RR-intervallet ikke større end ’rrMiss’ er en R-Peak ikke fundet, og der sker intet. Hvis RR-intervallet ikke er større end ’threshold1’, opdateres værdierne for ’npkf’, ’threshold1’ og ’threshold2’, og den fundne peak er heller ikke her en R-Peak.  Til sidst returnerer funktionen -1 hvis en R-Peak ikke blev detekteret og nummeret på R-Peak’en, hvis en R-Peak blev detekteret. Denne værdi returneres til ’findPeaks’-funktionen, som sender den videre til ’main’-funktionen, der på denne måde ved om en R-Peak er blevet fundet eller ej, og også hvilken plads i arrayet den ligger på, hvis det er tilfældet at en R-Peak blev fundet.
 
Værdierne til at detektere R-Peaks opdateres løbende, som peaks detekteres, men er i starten defineret ud fra gæt. Estimatet SPKF sættes i starten til 4000 vurderet ud fra de høje peaks værdier på figur x. NPKF vurderes til 1500 ud fra figur y, da der under denne værdi findes peaks, som bør klassificeres som støj. rrAvg1 og rrAvg2 sættes til 151, da dette omtrent er afstanden mellem to R-Peaks. Resten af værdierne: threshold1, threshold2, rrMiss, rrLow og rrHigh kan nu beregnes ud fra estimaterne.

I peaks.c anvendes statiske variabler, da det ikke ønskes, at variablerne skal reinitialiseres, hver gang funktionen kaldes. Da variablerne ikke anvendes uden for funktion, og deraf ikke skal gemmes flere gange, vil de ikke bruge mere hukommelse end nødvendigt hukommelse. Alternativt kan man initialiserer variablerne i main.c, og give variablerne som parametre til funktionen. Dette ville også bruge lidt hukommelse.

\subsection{Output}
Når en R-Peak er fundet meddeles den til patienten, som output sammen med tiden hvor den opstod. Denne beregnes ved at dele nummeret på målingen med 250, idet der måles 250 samples pr. sekund. Pulsen beregnes ved [ASK ANNA].
I ’peaks’-filen holder tælleren ’warningCounter’ ’øje med, om RR-intervallet ligger mellem ’rrLow’ og ’rrHigh’, da den i denne situation altid sættes til 0 uanset hvad den før skulle ligge på. Ligger RR-intervallet ikke mellem ’rrLow’ og ’rrHigh’, vokser den med værdien 1, og kommer den på et tidspunkt op på 5 – altså har den ikke været mellem ’rrLow’ og ’rrHigh’ fem gange i træk, gives en advarsel til patienten.
Hvis en fundet R-Peak har en værdi, der er mindre end 2000 gives også en advarsel.

\section{Resultater}

\subsection{Performance køretider}
Vi har lavet en gprof-profilering af vores færdige system, hvor det store datasæt (ECG10800K.txt) med 10,800,000 samples er blevet brugt, for at få bedst mulig profilering. 
Nedenfor ses en tabel med tidsfordelingen af de fire c-filer, og hver af deres funktioner.
På grund af computerens flere processorer, som alle bruger CPU’en, kører samtidigt, gør dette at resurserne til programmet varierer, og derved varierer profileringen også. Derfor har vi lavet 4 sæt profileringer, og tabellen nedenfor er gennemsnittet af de 4 sæt profileringer.

Tidsfordeling for c-filer
Main.c
0,070 s
Sensor.c
0,095 s
Filter.c
2,248 s
Peaks.c
0,070 s

Tidsfordeling for Filter-funktioner
lowPass
0,158 s
highPass
0,253 s
derivative
0,145 s
squaring
0,443 s
movingWindow
1,588 s

Tidsfordeling for Sensor-funktioner
getNextData
0,028 s
saveData
0,068 s

Tidsfordeling for Peaks-funktioner
findPeaks
0,068 s
findRPeaks
0,003 s

Total Time:
2,490 s

Der ses på diagrammet nedenfor en tidsfordeling af alle vores c-filer. Det ses, at Filter.c bruger 90% af CPU-tiden, når programmet bliver kørt. 

Da Filter.c bruger størstedelen af CPU-tiden, er det ret relevant at kigge på dens funktioner. Det ses, at Moving Window Integration-funktionen bruger 61 % af CPU-tiden for Filter.c. Hvis man skulle få programmet til at køre hurtigere, ville det nok være smart at effektivisere Moving Window Integration, da den bruger så stor en procentdel af tiden. Dette bliver diskuteret nærmere under diskussionsafsnittet.

\subsection{Filtrerede R-peaks}

\section{Diskussion}

\subsection{Detekterede R-peaks og advarsler}

\subsection{Effektivisering af kode}
Under resultatafsnittet sås det, at Moving Window Integration-funktionen brugte størstedelen af CPU-tiden, og det derfor kunne være ret relevant at effektivisere den. 
Det er muligt helt at undgå den for-løkke, der kører 30 gange i movingWindow. Hvis man i stedet for at skulle addere de seneste 30 x-værdier hver gang movingWindow bliver kaldt, kunne man bare gemme summen til næste sample, og bare subtraktere den ældste sample fra summen og addere den nyeste sample til summen. Dette kører i konstant tid frem for lineær tid, og hvis vi havde tid til at gøre dette i vores program, vil vi forvente en markant ændring i CPU-tiden.

\newpage
\section{Konklusion}
Det kan konkluderes, at det er lykkedes at fremstille et program, der indlæser simuleret data fra en tekstfil, filtrerer det og deraf bl.a. fjerner støj fra det og finder toppunkter i dataen, hvorefter det vurderes om der er tale om et hjerteslag. Dertil vurderer programmet, om hjerterytmen er ustabil, og viser data til brugeren 

\end{document}